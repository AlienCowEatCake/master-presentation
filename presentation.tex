\documentclass[aspectratio=43]{beamer}

% Русский язык (LaTeX)
\usepackage[T1,T2A]{fontenc}
\usepackage[utf8]{inputenc}
\usepackage[english,russian]{babel}

% Картинки
\usepackage{graphicx}
% Каталог для картинок
\graphicspath{ {./images/} }

% Разные таблицы
\usepackage{tabularx, multirow, tabu}
\newcolumntype{L}[1]{>{\hsize=#1\hsize\raggedright\arraybackslash}X}
\newcolumntype{R}[1]{>{\hsize=#1\hsize\raggedleft\arraybackslash}X}
\newcolumntype{C}[1]{>{\hsize=#1\hsize\centering\arraybackslash}X}

% Таблицы и рисунки по ГОСТу
\usepackage[tableposition=top, singlelinecheck=false]{caption}
\DeclareCaptionLabelFormat{gostfigure}{Рисунок #2}\DeclareCaptionLabelFormat{gosttable}{Таблица #2}
\DeclareCaptionLabelSeparator{gost}{~---~}
\captionsetup{labelsep=gost}
\captionsetup*[figure]{labelformat=gostfigure, justification=centering}
\captionsetup*[table]{labelformat=gosttable, justification=raggedright}
\usepackage{subfig}
\renewcommand{\thesubfigure}{\asbuk{subfigure}}
% в рисунках: h - "хотелось бы картинку здесь"; h! - "очень хочу картинку здесь"; H - "ХОЧУ картинку здесь и баста", p - на отдельной странице, t - сверху
% Управление плавающими штуковинами (рисунками, таблицами, etc)
\usepackage{float}

% Запрет висячих строк
\clubpenalty=10000
\widowpenalty=10000

% Для формул
\usepackage{amsmath}
\usepackage{mathtools}
\usepackage{amsfonts}
\usepackage{amssymb}
\usepackage{amsbsy}

% Удобные ссылки на электронные ресурсы
\usepackage{url}
\renewcommand\UrlFont{}

% Минимальное количество букв, которые можно переносить
\righthyphenmin=2

% Борьба с overfull
\tolerance=300000

% Перенос знаков во внутритекстовых формулах (использовать так $a\hm+b\hm+c\hm+d$)
%\newcommand*{\hm}[1]{#1\nobreak\discretionary{}{\hbox{$\mathsurround=0pt #1$}}{}}
% Можно также просто запретить переносить знаки операций и отношений
\binoppenalty=10000
\relpenalty=10000

%
\title{Анализ систем источник-приёмник в задачах морской геоэлектрики}
\author{Жигалов Петр Сергеевич}
\institute{Новосибирский государственный технический университет \newline
Факультет прикладной математики и информатики \newline
Кафедра вычислительных технологий}
\date{2016 г.}

% Команды и сокращения
\renewcommand{\Re}{\mathop{\mathrm{Re}}\nolimits}
\renewcommand{\Im}{\mathop{\mathrm{Im}}\nolimits}
\newcommand{\Jmp}[1]{[\![ { #1 } ]\!]}
\newcommand{\Avg}[1]{\{\!\!\{ { #1 } \}\!\!\}}
\newcommand{\Flux}[1]{\reallywidehat{ #1 } }
\newcommand{\CalTau}{\mathcal{T}}
\newcommand{\CalEps}{\mathcal{E}}
\newcommand{\CalF}{\mathcal{F}}
\renewcommand{\log}{\mathop{\mathrm{log}}\nolimits}
\newcommand{\CodeFont}[1]{{\small{\texttt{#1}}}}
\newcommand{\MakeTitle}[1]{\frametitle{\hspace{1.5em}\textbf{#1} \hfill \insertframenumber{} }}

% Тема
% https://www.hartwork.org/beamer-theme-matrix/
\usetheme{Rochester}
\usecolortheme{whale}

% =============================================================================

\begin{document}

\begin{frame}
	\titlepage
\end{frame}

% =============================================================================

\begin{frame}
	\MakeTitle{Цели и задачи}
	\textbf{Цель работы:} решение трёхмерной прямой задачи морской геоэлектрики векторным методом конечных элементов
	\newline\newline
	\textbf{Задачи:}
	\begin{itemize}
		\item Исследование влияния слоя воздуха при различной глубине источника электромагнитного возмущения
		\item Исследование целесообразности применения PML-слоя для ограничения области моделирования в задачах морской геоэлектрики на низких частотах
		\item Исследование поведения электромагнитного поля при различном расположении источника поля и искомого объекта друг относительно друга
	\end{itemize}
\end{frame}

% =============================================================================

\begin{frame}
	\MakeTitle{Задачи морской геоэлектрики}
	\begin{figure}[ht]
		\includegraphics[width=\textwidth,height=\textheight,keepaspectratio]{10000000000004B4000002D101BC1A4D.png}
	\end{figure}
\end{frame}

% =============================================================================

\begin{frame}
	\MakeTitle{Математическая модель}
	\textbf{Уравнение Гельмгольца:}
	\begin{equation}
		\nabla \times ( \mu^{-1} \nabla \times \mathbf{E} ) + k^{2} \mathbf{E} = - i \omega \mathbf{J} , \text{~~~} k^{2} = i \omega \sigma - \omega^{2} \varepsilon \label{eq:helmholtz}
	\end{equation}

	\begin{small}
	$\mathbf{E}$~--~напряжённость электрического поля~(В/м), \\
	$\sigma$~--~электрическая проводимость~(См/м), \\
	$\varepsilon = \varepsilon_r \varepsilon_0$~--~диэлектрическая проницаемость~(Ф/м), \\
	$\mu = \mu_r \mu_0$~--~магнитная проницаемость~(Гн/м), \\
	$\mathbf{J}$~--~плотность стороннего электрического тока~(А/м${}^2$).
	\end{small}
%	
	\newline\newline
	\textbf{Краевые условия:}
	\begin{equation}
		\left. \mathbf{E} \times \mathbf{n} \right | _{ S_1 } = {\mathbf{E}} ^g , \label{eq:bound1}
	\end{equation}
	\begin{equation}
		\left. \sigma \mathbf{E} \cdot \mathbf{n} \right | _{ S_2 } = 0 . \label{eq:bound2}
	\end{equation}
	
\end{frame}

% =============================================================================

\begin{frame}
	\MakeTitle{Вариационная постановка}
	\textbf{Пространства:}
	\begin{equation*}
		\mathbb{H} ( \mathrm{rot}\,, \Omega ) = \lbrace \mathbf{v} \in [\mathbb{L}^{2}(\Omega)]^{3} : \nabla \times \mathbf{v} \in [\mathbb{L}^{2}(\Omega)]^{3} \rbrace , \label{eq:H_rot}
	\end{equation*}
	\begin{equation*}
		\mathbb{H}_{0}( \mathrm{rot}\,, \Omega ) = \lbrace \mathbf{v} \in \mathbb{H}(\mathrm{rot}\,, \Omega) : \left. \mathbf{v} \times \mathbf{n} \right|_{\partial \Omega} = 0  \rbrace . \label{eq:H0_rot}
	\end{equation*}
%	
	\newline\newline
	\textbf{Вариационная постановка:} Найти $\mathbf{E} \in \mathbb{H}_{0}( \mathrm{rot}\,, \Omega )$, такое что $\forall \mathbf{v} \in \mathbb{H}_{0}( \mathrm{rot}\,, \Omega )$ будет выполнено:
	\begin{equation}
		\int\limits_\Omega \mu^{-1} \nabla \times \mathbf{E} \cdot \nabla \times \overline{\mathbf{v}} \,d\Omega + \int\limits_\Omega k^{2} \mathbf{E} \cdot \overline{\mathbf{v}} \,d\Omega = - \int\limits_\Omega i \omega \mathbf{J} \cdot \overline{\mathbf{v}} \,d\Omega . \label{eq:weak}
	\end{equation}
\end{frame}

% =============================================================================

\begin{frame}
	\MakeTitle{PML-слой}
	\textbf{Комплексное растяжение координат:}
	\begin{equation*}
		\tilde{x} = \int\limits_0^x s_x (t) \,dt ,
		\text{~~~~~}
		\tilde{y} = \int\limits_0^y s_y (t) \,dt ,
		\text{~~~~~}
		\tilde{z} = \int\limits_0^z s_z (t) \,dt ,
	\end{equation*}
	\begin{equation*}
		\begin{cases}
		\displaystyle
		s_j(\tau) = 1 \text{~~--~вне PML-слоя,} \\
		\displaystyle
		s_j(\tau) = 1 + \chi \left( \frac{d(\tau)}{\delta} \right)^m , \text{~~} m \geq 1 \text{~~--~внутри PML-слоя,}
		\label{eq:pml_s}
		\end{cases}
	\end{equation*}
	где $d(\tau)$ -- расстояние в $j$-м направлении от внутренней границы PML-слоя, $\delta$ -- толщина PML-слоя, $\chi$ -- некоторое комплексное число, причём $\Re(\chi) \ge 0$, $\Im(\chi) \ge 0$.
\end{frame}

% =============================================================================

\begin{frame}
	\MakeTitle{Вариационная постановка с PML-слоем}
	\textbf{Оператор $\nabla$ в новых координатах:}
	\begin{equation*}
		\tilde{\nabla} = \left[ \frac{1}{s_x} \frac{\partial}{\partial x} \,, \frac{1}{s_y} \frac{\partial}{\partial y} \,, \frac{1}{s_z} \frac{\partial}{\partial z} \right] .
	\end{equation*}

	\textbf{Вариационная постановка:}\newline Найти $\mathbf{E} \in \mathbb{H}_{0}( \mathrm{rot}\,, \widehat{\Omega} = \Omega \setminus {\Omega^{PML}} )$ и  $\tilde{\mathbf{E}} \in \mathbb{H}_{0}( \mathrm{rot}\,, {\Omega^{PML}} )$, такие что $\forall \mathbf{v} \in \mathbb{H}_{0}( \mathrm{rot}\,, \widehat{\Omega} )$ и $\forall \tilde{\mathbf{v}} \in \mathbb{H}_{0}( \mathrm{rot}\,, {\Omega^{PML}} )$ будет выполнено:
\begin{equation*}
	\begin{cases}
		\displaystyle
		\int\limits_{\widehat{\Omega}} \mu^{-1} \nabla \times \mathbf{E} \cdot \nabla \times \overline{\mathbf{v}} \,d\widehat{\Omega} + \int\limits_{\widehat{\Omega}} k^{2} \mathbf{E} \cdot \overline{\mathbf{v}} \,d\widehat{\Omega} = - \int\limits_{\widehat{\Omega}} i \omega \mathbf{J} \cdot \overline{\mathbf{v}} \,d\widehat{\Omega} \\
		\displaystyle
		\int\limits_{{\Omega^{PML}}} \mu^{-1} \tilde{\nabla} \times \tilde{\mathbf{E}} \cdot \tilde{\nabla} \times \tilde{\overline{\mathbf{v}}} \,d{\Omega^{PML}} + \int\limits_{{\Omega^{PML}}} k^{2} \tilde{\mathbf{E}} \cdot \tilde{\overline{\mathbf{v}}} \,d{\Omega^{PML}} = 0 . \\
	\end{cases}
\end{equation*}
\end{frame}

% =============================================================================

\begin{frame}
	\MakeTitle{Исследование влияния слоя воздуха}
	\textbf{Описание расчётной области}
	\begin{columns}[t,totalwidth=\linewidth]
		\begin{column}{.5\linewidth}
			\vspace{-3em}
			\begin{figure}[ht]
				\includegraphics[width=1.1\textwidth,height=1.1\textheight,keepaspectratio]{area_3layers_shift_3.eps}
			\end{figure}
    	\end{column}
    	\begin{column}{.04\linewidth}
    	\end{column}
		\begin{column}{.8\linewidth}
			\\
			\begin{small}
			$\Omega_1$ -- воздух ($\sigma=10^{-6}$ См/м); \\
			$\Omega_2$ -- морская вода ($\sigma=3.3$ См/м); \\
			$\Omega_3$ -- грунт ($\sigma=0.2$ См/м); \\
			$\Omega_4$ -- углеводороды ($\sigma=10^{-2}$~См/м); \\
			$L_1 = L_2 = L_3 = 6000$~м; \\
			$h_1=600$~м; $h_2=135$~м; \\
			$h_3=75$~м; $l_1=400$~м; \\
			$d=100$~м; $\nu=1$~Гц. \\
			$h$ варьируется в ходе исследования.
			\end{small}
		\end{column}
	\end{columns}
\end{frame}

% =============================================================================

\begin{frame}
	\MakeTitle{Исследование влияния слоя воздуха}
	\textbf{Относительная разность решений при изменении глубины петли $h$:}
	\begin{small}
	\begin{table}[ht]
		\begin{tabularx}{\textwidth}{|C{2.89}|C{0.73}|C{0.73}|C{0.73}|C{0.73}|C{0.73}|C{0.73}|C{0.73}|}
			\hline Глубина петли & 5 & 10 & 50 & 100 & 200 & 300 & 400 \\
			\hline $\displaystyle \frac{\| \mathbf{E}^{air} - \mathbf{E}^{noair} \|_{\mathbb{L}^2}}{\| \mathbf{E}^{air} \|_{\mathbb{L}^2}}$ & 0.44 & 0.40 & 0.24 & 0.14 & 0.07 & 0.04 & 0.02 \\
			\hline
		\end{tabularx}
	\end{table}
	\end{small}
	\vspace{-1em}
	\begin{figure}[ht]
		\includegraphics[scale=0.7]{presentation.eps}
	\label{fig:res1:graph}
	\end{figure}
\end{frame}

% =============================================================================

\begin{frame}
	\MakeTitle{Исследование влияния слоя воздуха}
	\textbf{$\Re(\mathbf{E}_y)$ по линии $y=0$, $z=-610$:}
	\begin{columns}[t,totalwidth=\linewidth]
		\begin{column}{.5\linewidth}
			\vspace{-1.7em}
			\begin{center}
			\includegraphics[width=\textwidth,height=0.4\textheight,keepaspectratio]{deep_-5.eps} \\
			\vspace{-0.1em}
			\tiny{$h=-5$} \\
			\includegraphics[width=\textwidth,height=0.4\textheight,keepaspectratio]{deep_-200.eps} \\
			\vspace{-0.1em}
			\tiny{$h=-200$} \\
			\end{center}
		\end{column}
		\begin{column}{.5\linewidth}
			\vspace{-1.7em}
			\begin{center}
			\includegraphics[width=\textwidth,height=0.4\textheight,keepaspectratio]{deep_-50.eps} \\
			\vspace{-0.1em}
			\tiny{$h=-50$} \\
			\includegraphics[width=\textwidth,height=0.4\textheight,keepaspectratio]{deep_-300.eps} \\
			\vspace{-0.1em}
			\tiny{$h=-300$} \\
			\end{center}
		\end{column}
	\end{columns}
\end{frame}

% =============================================================================

\begin{frame}
	\MakeTitle{Заключение}
	В работе были реализованы алгоритмы на базе векторного метода конечных элементов. Эти алгоритмы были положены в основу программного комплекса, который позволяет моделировать электромагнитные поля в областях разнообразной структуры. С помощью этого программного комплекса были решены модельные задачи морской геоэлектрики на низких частотах, проведены исследования возможности сокращения области моделирования без внесения дополнительных погрешностей.

	На основании исследований были сделаны выводы, что расчёты, в которых в область моделирования не включается воздух, допустимы только при расположении источника электромагнитного поля на большой глубине, иначе, из-за неправильного учёта физических процессов, происходящих в воздухе, полученное решение будет неверным.

	Применение PML-слоя позволило получить достаточно точные решения, однако его применение не привело к резкому уменьшению размерности систем уравнений и, как следствие, к уменьшению времени решения. Было выдвинуто предположение, что увеличить эффективность применения PML-слоя можно с помощью применения неконформных методов, в которых конечноэлементная сетка может содержать геометрические носители разного типа: тетраэдры и параллелепипеды.

	Расчёты на модельной задаче с проводящим и непроводящим объектами показали, что проводящий объект хорошо <<виден>> на некотором расстоянии от морского дна, а непроводящий -- только вблизи дна или при небольшом заглублении приёмника в грунт. Наибольший отклик на источник электромагнитного возмущения для непроводящего объекта наблюдался в том случае, когда источник располагался со смещением от центра симметрии объекта.
\end{frame}


% =============================================================================


\end{document}

